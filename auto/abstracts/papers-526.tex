Contextual influences on language often exhibit substantial cross-lingual regularities; for example, we are more verbose in situations that require finer distinctions. However, these regularities are sometimes obscured by semantic and syntactic differences. Using a newly-collected dataset of color reference games in Mandarin Chinese (which we release to the public), we confirm that a variety of constructions display the same sensitivity to contextual difficulty in Chinese and English. We then show that a neural speaker agent trained on bilingual data with a simple multitask learning approach displays more human-like patterns of context dependence and is more pragmatically informative than its monolingual Chinese counterpart. Moreover, this is not at the expense of language-specific semantic understanding: the resulting speaker model learns the different basic color term systems of English and Chinese (with noteworthy cross-lingual influences), and it can identify synonyms between the two languages using vector analogy operations on its output layer, despite having no exposure to parallel data.
