Hate speech detection is a critical, yet challenging problem in Natural Language Processing (NLP). Despite the existence of numerous studies dedicated to the development of NLP hate speech detection approaches, the accuracy is still poor. The central problem is that social media posts are short and noisy, and most existing hate speech detection solutions take each post as an isolated input instance, which is likely to yield high false positive and negative rates. In this paper, we radically improve automated hate speech detection by presenting a novel model that leverages intra-user and inter-user representation learning for robust hate speech detection on Twitter. In addition to the target Tweet, we collect and analyze the user's historical posts to model intra-user Tweet representations. To suppress the noise in a single Tweet, we also model the similar Tweets posted by all other users with reinforced inter-user representation learning techniques. Experimentally, we show that leveraging these two representations can significantly improve the f-score of a strong bidirectional LSTM baseline model by 10.1\%.
