Mental health forums are online spaces where people can share their experiences anonymously and get peer support. These forums, require the supervision of moderators to provide support in delicate cases, such as posts expressing suicide ideation. The large increase in the number of forum users makes the task of the moderators unmanageable without the help of automatic triage systems. In the present paper, we present a Machine Learning approach for the triage of posts. Most approaches in the literature focus on the content of the posts, but only a few authors take advantage of features extracted from the context in which they appear. Our approach consists of the development and implementation of a large variety of new features from both, the content and the context of posts, such as previous messages, interaction with other users and author's history. Our method has competed in the CLPsych 2017 Shared Task, obtaining the first place for several of the subtasks. Moreover, we also found that models that take advantage of post context improve significantly its performance in the detection of flagged posts (posts that require moderators attention), as well as those that focus on post content outperforms in the detection of most urgent events.
