Frame induction is the automatic creation of frame-semantic resources similar to FrameNet or PropBank, which map lexical units of a language to frame representations of each lexical unit's semantics. For verbs, these representations usually include a specification of their argument slots and of the selectional restrictions that apply to each slot. Verbs that participate in diathesis alternations have different syntactic realizations whose semantics are closely related, but not identical. We discuss the influence that such alternations have on frame induction, compare several possible frame structures for verbs in the causative alternation, and propose a systematic analysis of alternating verbs that encodes their similarities as well as their differences.
