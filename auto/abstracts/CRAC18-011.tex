This paper describes the design and evaluation of a system for the automatic detection and resolution of shell nouns in German. Shell nouns are general nouns, such as fact, question, or problem, whose full interpretation relies on a content phrase located elsewhere in a text, which these nouns simultaneously serve to characterize and encapsulate. To accomplish this, the system uses a series of lexico-syntactic patterns in order to extract shell noun candidates and their content in parallel. Each pattern has its own classifier, which makes the final decision as to whether or not a link is to be established and the shell noun resolved. Overall, about 26.2\% of the annotated shell noun instances were correctly identified by the system, and of these cases, about 72.5\% are assigned the correct content phrase. Though it remains difficult to identify shell noun instances reliably (recall is accordingly low in this regard), this system usually assigns the right content to correctly classified cases. cases.
