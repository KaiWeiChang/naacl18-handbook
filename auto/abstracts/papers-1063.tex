Alzheimer's disease (AD) is an irreversible and progressive brain disease that can be stopped or slowed down with medical treatment. Language changes serve as a sign that a patient's cognitive functions have been impacted, potentially leading to early diagnosis. In this work, we use NLP techniques to classify and analyze the linguistic characteristics of AD patients using the DementiaBank dataset. We apply three neural models based on CNNs, LSTM-RNNs, and their combination, to distinguish between language samples from AD and control patients. We achieve a new independent benchmark accuracy for the AD classification task. More importantly, we next interpret what these neural models have learned about the linguistic characteristics of AD patients, via analysis based on activation clustering and first-derivative saliency techniques. We then perform novel automatic pattern discovery inside activation clusters, and consolidate AD patients' distinctive grammar patterns. Additionally, we show that first derivative saliency can not only rediscover previous language patterns of AD patients, but also shed light on the limitations of neural models. Lastly, we also include analysis of gender-separated AD data.
