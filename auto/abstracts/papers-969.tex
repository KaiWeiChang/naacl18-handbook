Large-scale datasets for natural language inference are created by presenting crowd workers with a sentence (premise), and asking them to generate three new sentences (hypotheses) that it entails, contradicts, or is logically neutral with respect to. We show that, in a significant portion of such data, this protocol leaves clues that make it possible to identify the label by looking only at the hypothesis, without observing the premise. Specifically, we show that a simple text categorization model can correctly classify the hypothesis alone in about 67\% of SNLI (Bowman et. al, 2015) and 53\% of MultiNLI (Williams et. al, 2017). Our analysis reveals that specific linguistic phenomena such as negation and vagueness are highly correlated with certain inference classes. Our findings suggest that the success of natural language inference models to date has been overestimated, and that the task remains a hard open problem.
