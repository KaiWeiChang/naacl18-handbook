A natural language argument is composed of a claim as well as reasons given as premises for the claim. The warrant explaining the reasoning is usually left implicit, as it is clear from the context and common sense. This makes a comprehension of arguments easy for humans but hard for machines. This paper summarizes the first shared task on argument reasoning comprehension. Given a premise and a claim along with some topic information, the goal was to automatically identify the correct warrant among two candidates that are plausible and lexically close, but in fact imply opposite claims. We describe the dataset with 1970 instances that we built for the task, and we outline the 21 computational approaches that participated, most of which used neural networks. The results reveal the complexity of the task, with many approaches hardly improving over the random accuracy of about 0.5. Still, the best observed accuracy (0.712) underlines the principle feasibility of identifying warrants. Our analysis indicates that an inclusion of external knowledge is key to reasoning comprehension.
