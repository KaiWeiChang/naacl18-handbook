Recognizing lexical semantic relations between word pairs is an important task for many applications of natural language processing. One of the mainstream approaches to this task is to exploit the lexico-syntactic paths connecting two target words, which reflect the semantic relations of word pairs. However, this method requires that the considered words co-occur in a sentence. This requirement is hardly satisfied because of Zipf's law, which states that most content words occur very rarely. In this paper, we propose novel methods with a neural model of P(path|w1,w2) to solve this problem. Our proposed model of P (path|w1, w2 ) can be learned in an unsupervised manner and can generalize the co-occurrences of word pairs and dependency paths. This model can be used to augment the path data of word pairs that do not co-occur in the corpus, and extract features capturing relational information from word pairs. Our experimental results demonstrate that our methods improve on previous neural approaches based on dependency paths and successfully solve the focused problem.
