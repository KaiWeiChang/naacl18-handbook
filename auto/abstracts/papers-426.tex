An essential step in FrameNet Semantic Role Labeling is the Frame Identification (FrameId) task, which aims at disambiguating a situation around a predicate. Whilst current FrameId methods rely on textual representations only, we hypothesize that FrameId can profit from a richer understanding of the situational context. Such contextual information can be obtained from common sense knowledge, which is more present in images than in text. In this paper, we extend a state-of-the-art FrameId system in order to effectively leverage multimodal representations. We conduct a comprehensive evaluation on the English FrameNet and its German counterpart SALSA. Our analysis shows that for the German data, textual representations are still competitive with multimodal ones. However on the English data, our multimodal FrameId approach outperforms its unimodal counterpart, setting a new state of the art. Its benefits are particularly apparent in dealing with ambiguous and rare instances, the main source of errors of current systems. For research purposes, we release (a) the implementation of our system, (b) our evaluation splits for SALSA 2.0, and (c) the embeddings for synsets and IMAGINED words.
