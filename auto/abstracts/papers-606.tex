Predicting the emotional value of lexical items is a well-known problem in sentiment analysis. While research has focused on polarity for quite a long time, meanwhile this early focus has been shifted to more expressive emotion representation models (such as Basic Emotions or Valence-Arousal-Dominance). This change resulted in a proliferation of heterogeneous formats and, in parallel, often small-sized, non-interoperable resources (lexicons and corpus annotations). In particular, the limitations in size hampered the application of deep learning methods in this area because they typically require large amounts of input data. We here present a solution to get around this language data bottleneck by rephrasing word emotion induction as a multi-task learning problem. In this approach, the prediction of each independent emotion dimension is considered as an individual task and hidden layers are shared between these dimensions. We investigate whether multi-task learning is more advantageous than single-task learning for emotion prediction  by comparing our model against a wide range of alternative emotion and polarity induction methods featuring 9 typologically diverse languages and a total of 15 conditions. Our model turns out to outperform each one of them. Against all odds, the proposed deep learning approach yields the largest gain on the smallest data sets, merely composed of one thousand samples.
