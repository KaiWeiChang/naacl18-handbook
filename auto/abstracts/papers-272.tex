Visual question answering (Visual QA) has attracted a lot of attention lately, seen essentially as a form of (visual) Turing test that artificial intelligence should strive to achieve. In this paper, we study a crucial component of this task: how can we design good datasets for the task?  We focus on the design of multiple-choice based datasets where the learner has to select the right answer from a set of candidate ones including the target (i.e., the correct one) and the decoys (i.e., the incorrect ones). Through careful analysis of the results attained by state-of-the-art learning models and human annotators on existing datasets, we show that the design of the decoy answers has a significant impact on how and what the learning models learn from the datasets. In particular, the resulting learner can ignore the visual information, the question, or both while still doing well on the task. Inspired by this, we propose automatic procedures  to remedy such design deficiencies. We apply the procedures to re-construct decoy answers for two popular Visual QA datasets as well as to create a new Visual QA dataset from the Visual Genome project, resulting in the largest dataset for this task. Extensive empirical studies show that the design deficiencies have been alleviated in the remedied datasets and the performance on them is likely a more faithful indicator of the difference among learning models. The datasets are released and publicly available via http://www.teds.usc.edu/website\_vqa/.
