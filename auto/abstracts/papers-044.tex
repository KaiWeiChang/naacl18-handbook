What makes some types of languages more probable than others? For instance, we know that almost all spoken languages contain the vowel phoneme /i/; why should that be? The field of linguistic typology seeks to answer these questions and, thereby, divine the mechanisms that underlie human language. In our work, we tackle the problem of vowel system typology, i.e., we propose a generative probability model of which vowels a language contains. In contrast to previous work, we work directly with the acoustic information---the first two formant values---rather than modeling discrete sets of symbols from the international phonetic alphabet. We develop a novel generative probability model and report results on over 200 languages.
