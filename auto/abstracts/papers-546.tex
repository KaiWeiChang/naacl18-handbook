Public debate forums provide a common platform for exchanging opinions on a topic of interest. While recent studies in natural language processing (NLP) have provided empirical evidence that the language of the debaters and their patterns of interaction play a key role in changing the mind of a reader, research in psychology has shown that prior beliefs can affect our interpretation of an argument and could therefore constitute a competing alternative explanation for resistance to changing one's stance. To study the actual effect of language use vs. prior beliefs on persuasion, we provide a new dataset and propose a controlled setting that takes into consideration two reader-level factors: political and religious ideology. We find that prior beliefs affected by these reader-level factors play a more important role than language use effects and argue that it is important to account for them in NLP studies of persuasion.
