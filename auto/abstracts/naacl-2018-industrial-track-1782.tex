Query auto completion (QAC) systems are a standard part of search engines in industry, helping users formulate their query. Such systems update their suggestions after the user types each character, predicting the user's intent using various signals --- one of the most common being popularity. Recently, deep learning approaches have been proposed for the QAC task, to specifically address the main limitation of previous popularity-based methods: the inability to predict unseen queries. In this work we improve previous methods based on neural language modeling, with the goal of building an end-to-end system. We particularly focus on using real-world data by integrating user information for personalized suggestions when possible. We also make use of time information and study how to increase diversity in the suggestions while studying the impact on scalability. Our empirical results demonstrate a marked improvement on two separate datasets over previous best methods in both accuracy and scalability, making a step towards neural query auto-completion in production search engines.
