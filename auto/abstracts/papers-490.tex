The task of automatic text summarization is to generate a short text that summarizes the most important information in a given set of documents. Sentence regression is an emerging branch in automatic text summarizations. Its key idea is to estimate the importance of information via learned utility scores for individual sentences. These scores are then used for selecting sentences from the source documents, typically according to a greedy selection strategy. Recently proposed state-of-the-art models learn to predict ROUGE recall scores of individual sentences, which seems reasonable since the final summaries are evaluated according to ROUGE recall. In this paper, we show in extensive experiments that following this intuition leads to suboptimal results and that learning to predict ROUGE precision scores leads to better results. The crucial difference is to aim not at covering as much information as possible but at wasting as little space as possible in every greedy step.
