Notional anaphors are pronouns which disagree with their antecedents' grammatical categories for notional reasons, such as plural to singular agreement in: ``the government ... they''. Since such cases are rare and conflict with evidence from strictly agreeing cases (''the government ... it''), they present a substantial challenge to both coreference resolution and referring expression generation. Using the OntoNotes corpus, this paper takes an ensemble approach to predicting English notional anaphora in context on the basis of the largest empirical data to date. In addition to state of the art prediction accuracy, the results suggest that theoretical approaches positing a plural construal at the antecedent's utterance are insufficient, and that circumstances at the anaphor's utterance location, as well as global factors such as genre, have a strong effect on the choice of referring expression.
