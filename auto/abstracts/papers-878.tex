Interacting with relational databases through natural language helps users with any background easily query and analyze a vast amount of data. This requires a system that understands users' questions and converts them to SQL queries automatically. In this paper, we present a novel approach TypeSQL which formats the problem as a slot filling task in a more reasonable way. In addition, TypeSQL utilizes type information to better understand rare entities and numbers in the questions. We experiment this idea on the WikiSQL dataset and outperform the prior art by 6\% in much shorter time. We also show that accessing the content of databases can significantly improve the performance when users' queries are not well-formed. TypeSQL can reach 82.6\% accuracy, a 17.5\% absolute improvement compared to the previous content-sensitive model.
