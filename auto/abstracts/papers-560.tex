Speakers often have more than one way to express the same meaning. What general principles govern speaker choice in the face of optionality when near semantically invariant alternation exists? Studies have shown that optional reduction in language is sensitive to contextual predictability, such that more predictable a linguistic unit is, the more likely it is to get reduced. Yet it is unclear whether these cases of speaker choice are driven by audience design versus toward facilitating production. Here we argue that for a different optionality phenomenon, namely classifier choice in Mandarin Chinese, Uniform Information Density and at least one plausible variant of availability-based production make opposite predictions regarding the relationship between the predictability of the upcoming material and speaker choices. In a corpus analysis of Mandarin Chinese, we show that the distribution of speaker choices supports the availability-based production account and not the Uniform Information Density.
