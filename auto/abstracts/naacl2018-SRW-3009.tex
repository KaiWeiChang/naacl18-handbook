Igbo is a low-resource language spoken by approximately 30 million people worldwide. It is the native language of the Igbo people of south-eastern Nigeria. In Igbo language, diacritics - orthographic and tonal - play a huge role in the distinguishing the meaning and pronunciation of words. Omitting diacritics in texts often leads to lexical ambiguity. Diacritic restoration is a pre-processing task that replaces missing diacritics on words from which they have been removed. In this work, we applied embedding models to the diacritic restoration task and compared their performances to those of n-gram models. Although word embedding models have been successfully applied to various NLP tasks, it has not been used, to our knowledge, for diacritic restoration. Two classes of word embeddings models were used: those projected from the English embedding space; and those trained with Igbo bible corpus ($\approx$ 1m). Our best result, 82.49\%, is an improvement on the baseline n-gram models.
