The use of explicit object detectors as an intermediate step to image captioning -- which used to constitute an essential stage in early work -- is often bypassed in the currently dominant end-to-end approaches, where the language model is conditioned directly on a mid-level image embedding. We argue that explicit detections provide rich semantic information, and can thus be used as an interpretable representation to better understand why end-to-end image captioning systems work well. We provide an in-depth analysis of end-to-end image captioning by exploring a variety of cues that can be derived from such object detections. Our study reveals that end-to-end image captioning systems rely on matching image representations to generate captions, and that encoding the frequency, size and position of objects are complementary and all play a role in forming a good image representation. It also reveals that different object categories contribute in different ways towards image captioning.
