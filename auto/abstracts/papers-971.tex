While Wikipedia exists in 287 languages, its content is unevenly distributed among them. In this work, we investigate the generation of open domain Wikipedia summaries in underserved languages using structured data from Wikidata. To this end, we propose a neural network architecture equipped with copy actions that learns to generate single-sentence and comprehensible textual summaries from Wikidata triples. We demonstrate the effectiveness of the proposed approach by evaluating it against a set of baselines on two languages of different natures: Arabic, a morphological rich language with a larger vocabulary than English, and Esperanto, a constructed language known for its easy acquisition.
