As computers and information grow a more integral part of our world, it is becoming more and more important for humans to be able to interact with their computers in complex ways. One way to do so is by programming, but the ability to understand and generate programming languages is a highly specialized skill. As a result, in the past several years there has been an increasing research interest in methods that focus on the intersection of programming and natural language, allowing users to use natural language to interact with computers in the complex ways that programs allow us to do. In this tutorial, we will focus on machine learning models of programs and natural language focused on making this goal a reality. First, we will discuss the similarities and differences between programming and natural language. Then we will discuss methods that have been designed to cover a variety of tasks in this field, including automatic explanation of programs in natural language (code-to-language), automatic generation of programs from natural language specifications (language-to-code), modeling the natural language elements of source code, and analysis of communication in collaborative programming communities. The tutorial will be aimed at NLP researchers and practitioners, aiming to describe the interesting opportunities that models at the intersection of natural and programming languages provide, and also how their techniques could provide benefit to the practice of software engineering as a whole.
