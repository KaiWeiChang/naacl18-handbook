In this paper we analyze the use of emojis in social media with respect to gender and skin tone. By gathering a dataset of over twenty two million tweets from United States some findings are clearly highlighted after performing a simple frequency-based analysis. Moreover, we carry out a semantic analysis on the usage of emojis and their modifiers (e.g. gender and skin tone) by embedding all words, emojis and modifiers into the same vector space. Our analyses reveal that some stereotypes related to the skin color and gender seem to be reflected on the use of these modifiers. For example, emojis representing hand gestures are more widely utilized with lighter skin tones, and the usage across skin tones differs significantly. At the same time, the vector corresponding to the male modifier tends to be semantically close to emojis related to business or technology, whereas their female counterparts appear closer to emojis about love or makeup.
