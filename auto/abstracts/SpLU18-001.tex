Prior methodologies for understanding spatial language have treated literal expressions such as ``Mary pushed the car over the edge'' differently from metaphorical extensions such as ``Mary's job pushed her over the edge''.  We demonstrate a methodology for standardizing literal and metaphorical meanings, by building on work in Lexical Conceptual Structure (LCS), a general-purpose representational component used in machine translation.  We argue that spatial predicates naturally extend into other fields (e.g., circumstantial or temporal), and that LCS provides both a framework for distinguishing spatial from non-spatial, and a system for finding metaphorical meaning extensions. We start with MetaNet (MN), a large repository of conceptual metaphors, condensing 197 spatial entries into sixteen top-level categories of motion frames.  Using naturally occurring instances of English push , and expansions of MN frames, we demonstrate that literal and metaphorical extensions exhibit patterns predicted and represented by the LCS model.
