This paper describes our system, entitled IronyMagnet, for the 3rd Task of the SemEval 2018 workshop, ``Irony Detection in English Tweets''. In Task 1, irony classification task has been considered as a binary classification task. Now for the first time, finer categories of irony are considered as part of a shared task. In task 2, three types of irony are considered; ``Irony by contrast'' - ironic instances where evaluative expression portrays inverse polar- ity (positive, negative) of the literal propo- sition; ``Situational irony'' - ironic instances where output of a situation do not comply with its expectation; ``Other verbal irony'' - in- stances where ironic intent does not rely on polarity contrast or unexpected outcome. We proposed a Siamese neural network for irony detection, which is consisted of two subnet- works, each containing a long short term mem- ory layer(LSTM) and an embedding layer ini- tialized with vectors from Glove word embed- ding 1 . The system achieved a f-score of 0.72, and 0.50 in task 1, and task 2 respectively.
