Question Answering for complex questions is often modelled as a graph construction or traversal task, where a solver must build or traverse a graph of facts that answer and explain a given question.  This ``multi-hop'' inference has been shown to be extremely challenging, with few models able to aggregate more than two facts before being overwhelmed by ``semantic drift'', or the tendency for long chains of facts to quickly drift off topic.  This is a major barrier to current inference models, as even elementary science questions require an average of 4 to 6 facts to answer and explain. In this work we empirically characterize the difficulty of building or traversing a graph of sentences connected by lexical overlap, by evaluating chance sentence aggregation quality through 9,784 manually-annotated judgements across knowledge graphs built from three free-text corpora (including study guides and Simple Wikipedia). We demonstrate semantic drift tends to be high and aggregation quality low, at between 0.04 and 3, and highlight scenarios that maximize the likelihood of meaningfully combining information.
