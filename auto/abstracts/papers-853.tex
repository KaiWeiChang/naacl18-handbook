In this paper we explore the use of Learning Hidden Unit Contribution for the task of neural machine translation. The method was initially proposed in the context of speech recognition for adapting a general system to the specific acoustic characteristics of each speaker. Similar in spirit, in a machine translation framework we want to adapt a general system to a specific domain. We show that the proposed method achieves improvements of up to 2.6 BLEU points over a general system, and up to 6 BLEU points if the initial system has only been trained on out-of-domain data, a situation which may easily happen in practice.  The good performance together with its short training time and small memory footprint make it a very attractive solution for domain adaptation.
