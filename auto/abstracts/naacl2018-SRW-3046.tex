Conversation is a joint social process, with participants cooperating to exchange information. This process is helped along through linguistic alignment: participants' adoption of each other's word use. This alignment is robust, appearing many settings, and is nearly always positive. We create an alignment model for examining alignment in Twitter conversations across antagonistic groups. This model finds that some word categories, specifically pronouns used to establish group identity and common ground, are negatively aligned. This negative alignment is observed despite other categories, which are less related to the group dynamics, showing the standard positive alignment.  This suggests that alignment is strongly biased toward cooperative alignment, but that different linguistic features can show substantially different behaviors.
