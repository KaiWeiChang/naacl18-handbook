While labor issues and quality assurance in crowdwork are increasingly studied, how annotators make sense of texts and how they are personally impacted by doing so are not. We study these questions via a narrative-sorting annotation task, where carefully selected (by sequentiality, topic, emotional content, and length) collections of tweets serve as examples of everyday storytelling. As readers process these narratives, we measure their facial expressions, galvanic skin response, and self-reported reactions. From the perspective of annotator well-being, a reassuring outcome was that the sorting task did not cause a measurable stress response, however readers reacted to humor. In terms of sensemaking, readers were more confident when sorting sequential, target-topical, and highly emotional tweets. As crowdsourcing becomes more common, this research sheds light onto the perceptive capabilities and emotional impact of human readers.
