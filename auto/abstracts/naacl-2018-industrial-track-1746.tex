In large-scale educational assessments, the use of automated scoring has recently become quite common. While the majority of student responses can be processed and scored without difficulty, there are a small number of responses that have atypical characteristics that make it difficult for an automated scoring system to assign a correct score. We describe a pipeline that detects and processes these kinds of responses at run-time. We present the most frequent kinds of what are called non-scorable responses along with effective filtering models based on various NLP and speech processing technologies. We give an overview of two operational automated scoring systems ---one for essay scoring and one for speech scoring--- and describe the filtering models they use. Finally, we present an evaluation and analysis of filtering models used for spoken responses in an assessment of language proficiency.
