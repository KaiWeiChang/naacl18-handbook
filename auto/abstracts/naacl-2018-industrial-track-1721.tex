Spoken Language Understanding (SLU), which extracts semantic information from speech, is not flawless, specially in practical applications. The reliability of the output of an SLU system can be evaluated using a semantic confidence measure. Confidence measures are a solution to improve the quality of spoken dialogue systems, by rejecting low-confidence SLU results. In this study we discuss real-world applications of confidence scoring in a customer service scenario. We build confidence models for three major types of dialogue states that are considered as different domains: how may I help you, number capture, and confirmation. Practical challenges to train domain-dependent confidence models, including data limitations, are discussed, and it is shown that feature engineering plays an important role to improve performance.  We explore a wide variety of predictor features based on speech recognition, intent classification, and high-level domain knowledge, and find the combined feature set with the best rejection performance for each application.
