Reinforcement learning (RL) is a promising approach to solve dialogue policy optimisation. Traditional RL algorithms, however, fail to scale to large domains due to the curse of dimensionality. We propose a novel Dialogue Management architecture, based on Feudal RL, which decomposes the decision into two steps; a first step where a master policy selects a subset of primitive actions, and a second step where a primitive action is chosen from the selected subset. The structural information included in the domain ontology is used to abstract the dialogue state space, taking the decisions at each step using different parts of the abstracted state. This, combined with an information sharing mechanism between slots, increases the scalability to large domains. We show that an implementation of this approach, based on Deep-Q Networks, significantly outperforms previous state of the art in several dialogue domains and environments, without the need of any additional reward signal.
