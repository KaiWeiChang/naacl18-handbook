In this paper we introduce a new publicly available dataset for verification against textual sources, FEVER: Fact Extraction and VERification. It consists of 185,445 claims generated by altering sentences extracted from Wikipedia and subsequently verified without knowledge of the sentence they were derived from. The claims are classified as Supported, Refuted or NotEnoughInfo by annotators achieving 0.6841 in Fleiss kappa.  For the first two classes, the annotators also recorded the sentence(s) forming the necessary evidence for their judgment. To characterize the challenge of the dataset presented, we develop a pipeline approach and compare it to suitably designed oracles. The best accuracy we achieve on labeling a claim accompanied by the correct evidence is 31.87\%, while if we ignore the evidence we achieve 50.91\%. Thus we believe that FEVER is a challenging testbed that will help stimulate progress on claim verification against textual sources.
