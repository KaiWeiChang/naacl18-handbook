Trustfulness — one's general tendency to have confidence in unknown people or situations — predicts many important real-world outcomes such as mental health and likelihood to cooperate with others such as clinicians. While data-driven measures  of interpersonal  trust have previously been introduced, here, we develop the first language-based assessment of the personality trait  of trustfulness by fitting one's language to an accepted questionnaire-based trust score. Further, using trustfulness as a type of case study, we explore the role of questionnaire size  as  well as word count in developing language-based predictive models of users' psychological traits. We find that leveraging a longer questionnaire can yield greater test set accuracy, while, for training, we find it beneficial to include users who took smaller questionnaires which offers more observations for training. Similarly, after noting a decrease in individual prediction error as word count increased, we found a word  count-weighted training scheme was helpful when there were very few users in the first place.
