Temporal orientation refers to an individual's tendency to connect to the psychological concepts of past, present or future, and it affects personality, motivation, emotion, decision making and stress coping processes. The study of the social media users' psycho-demographic attributes from the perspective of human temporal orientation can be of utmost interest and importance to the business and administrative decision makers as it can provide an extra precious information for them to make informed decisions. In this paper, we propose a very first study to demonstrate the association between the sentiment view of the temporal orientation of the users and their different psycho-demographic attributes by analyzing their tweets. We first create a temporal orientation classifier in a minimally supervised way which classifies each tweet of the users in one of the three temporal categories, namely past, present, and future. A deep Bi-directional Long Short Term Memory (BLSTM) is used for the tweet classification task. Our tweet classifier achieves an accuracy of 78.27\% when tested on a manually created test set. We then determine the users' overall temporal orientation based on their tweets on the social media. The sentiment is added to the tweets at the fine-grained level where each temporal tweet is given a sentiment with either of the positive, negative or neutral. Our experiment reveals that depending upon the sentiment view of temporal orientation, a user's attributes vary. We finally measure the correlation between the users' sentiment view of temporal orientation and their different psycho-demographic factors using regression.
