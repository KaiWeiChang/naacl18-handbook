The four sub-tasks of SecureNLP build towards a capability for quickly highlighting critical information from malware reports, such as the specific actions taken by a  malware sample. Digital Operatives (DO) submitted to sub-tasks 1 and 2, using standard text analysis technology (text classification for sub-task 1, and a CRF for sub-task 2). Performance is broadly competitive with other submitted systems on sub-task 1 and weak on sub-task 2. The annotation guidelines for the intermediate sub-tasks create a linkage to the final task, which is both an annotation challenge and a potentially useful feature of the task. The methods that DO chose do not attempt to make use of this linkage, which may be a missed opportunity. This motivates a post-hoc error analysis. It appears that the annotation task is very hard, and that in some cases both deep conceptual knowledge and substantial surrounding context are needed in order to correctly classify sentences.
