As language technologies have become increasingly prevalent, there is a growing awareness that decisions we make about our data, methods, and tools are often tied up with their impact on people and societies. This tutorial will provide an overview of real-world applications of language technologies and the potential ethical implications associated with them. We will discuss philosophical foundations of ethical research along with state of the art techniques. Through this tutorial, we intend to provide the NLP researcher with an overview of tools to ensure that the data, algorithms, and models that they build are socially responsible. These tools will include a checklist of common pitfalls that one should avoid (e.g., demographic bias in data collection), as well as methods to adequately mitigate these issues (e.g., adjusting sampling rates or de-biasing through regularization). The tutorial is based on a new course on Ethics and NLP developed at Carnegie Mellon University.
