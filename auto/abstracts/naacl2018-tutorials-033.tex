Spoken Dialogue Systems (SDS) have great commercial potential as they promise to revolutionise the way in which humans interact with machines. The advent of deep learning led to substantial developments in this area of NLP research, and the goal of this tutorial is to familiarise the research community with the recent advances in what some call the most difficult problem in NLP. From a research perspective, the design of spoken dialogue systems provides a number of significant challenges, as these systems depend on: a) solving several difficult NLP and decision-making tasks; and b) combining these into a functional dialogue system pipeline. A key long-term goal of dialogue system research is to enable open-domain  systems that can converse about arbitrary topics and assist humans with completing a wide range of tasks. Furthermore, such systems need to autonomously learn on-line to improve their performance and recover from errors using both signals from their environment and from implicit and explicit user feedback. While the design of such systems has traditionally been modular, domain and language-specific, advances in deep learning have alleviated many of the design problems. The main purpose of this tutorial is to encourage dialogue research in the NLP community by providing the research background, a survey of available resources, and giving key insights to application of state-of-the-art SDS methodology into industry-scale conversational AI systems. We plan to introduce researchers to the pipeline framework for modelling goal-oriented dialogue systems, which includes three key components: 1) Language Understanding; 2) Dialogue Management; and 3) Language Generation. The differences between goal-oriented dialogue systems and chat-bot style conversational agents will be explained in order to show the motivation behind the design of both, with the main focus on the pipeline SDS framework. For each key component, we will define the research problem, provide a brief literature review and introduce the current state-of-the-art approaches. Complementary resources (e.g. available datasets and toolkits) will also be discussed. Finally, future work, outstanding challenges, and current industry practices will be presented. All of the presented material will be made available online for future reference.
