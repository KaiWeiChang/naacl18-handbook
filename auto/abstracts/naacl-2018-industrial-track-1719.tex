This paper investigates the use of Machine Translation (MT) to bootstrap a Natural Language Understanding (NLU) system for a new language for the use case of a large-scale voice-controlled device. The goal is to decrease the cost and time needed to get an annotated corpus for the new language, while still having a large enough coverage of user requests. Different methods of filtering MT data in order to keep utterances that improve NLU performance and language-specific post-processing methods are investigated. These methods are tested in a large-scale NLU task with translating around 10 millions training utterances from English to German. The results show a large improvement for using MT data over a grammar-based and over an in-house data collection baseline, while reducing the manual effort greatly. Both filtering and post-processing approaches improve results further.
