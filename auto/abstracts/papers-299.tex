In this paper, we propose a generalizable dialog generation approach that adapts multi-turn reasoning, one recent advancement in the field of document comprehension, to generate responses (''answers'') by taking current conversation session context as a ``document'' and current query as a ``question''. The major idea is to represent a conversation session into memories upon which attention-based memory reading mechanism can be performed multiple times, so that (1) user's query is properly extended by contextual clues and (2) optimal responses are step-by-step generated. Considering that the speakers of one conversation are not limited to be one, we separate the single memory used for document comprehension into different groups for speaker-specific topic and opinion embedding. Namely, we utilize the queries' memory, the responses' memory, and their unified memory, following the time sequence of the conversation session. Experiments on Japanese 10-sentence (5-round) conversation modeling show impressive results on how multi-turn reasoning can produce more diverse and acceptable responses than state-of-the-art single-turn and non-reasoning baselines.
