We present an empirical analysis of state-of-the-art systems for referring expression recognition -- the task of identifying the object in an image referred to by a natural language expression -- with the goal of gaining insight into how these systems reason about language and vision. Surprisingly, we find strong evidence that even sophisticated and linguistically-motivated models for this task may ignore linguistic structure, instead relying on shallow correlations introduced by unintended biases in the data selection and annotation process. For example, we show that a system trained and tested on the input image without the input referring expression can achieve a precision of 71.2\% in top-2 predictions. Furthermore, a system that predicts only the object category given the input can achieve a precision of 84.2\% in top-2 predictions. These surprisingly positive results for what should be deficient prediction scenarios suggest that careful analysis of what our models are learning -- and further, how our data is constructed -- is critical as we seek to make substantive progress on grounded language tasks.
