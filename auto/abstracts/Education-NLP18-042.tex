Given that all users of a language can be creative in their language usage, the overarching goal of this work is to investigate issues of variability and acceptability in written text, for both non-native speakers (NNSs) and native speakers (NSs).  We control for meaning by collecting a dataset of picture description task (PDT) responses from a number of NSs and NNSs, and we define and annotate a handful of features pertaining to form and meaning, to capture the multi-dimensional ways in which responses can vary and can be acceptable.  By examining the decisions made in this corpus development, we highlight the questions facing anyone working with learner language properties like variability, acceptability and native-likeness.  We find reliable inter-annotator agreement, though disagreements point to difficult areas for establishing a link between form and meaning.
