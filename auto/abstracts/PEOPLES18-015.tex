Recent studies in the field of text-based personality recognition experiment with different languages, feature extraction techniques, and machine learning algorithms to create better and more accurate models; however, little focus is placed on exploring the language use of a group of individuals defined by nationality. Individuals of the same nationality share certain practices and communicate certain ideas that can become embedded into their natural language. Many nationals are also not limited to speaking just one language, such as how Filipinos speak Filipino and English, the two national languages of the Philippines. The addition of several regional/indigenous languages, along with the commonness of code-switching, allow for a Filipino to have a rich vocabulary. This presents an opportunity to create a text-based personality model based on how Filipinos speak, regardless of the language they use. To do so, data was collected from 250 Filipino Twitter users. Different combinations of data processing techniques were experimented upon to create personality models for each of the Big Five. The results for both regression and classification show that Conscientiousness is consistently the easiest trait to model, followed by Extraversion. Classification models for Agreeableness and Neuroticism had subpar performances, but performed better than those of Openness. An analysis on personality trait score representation showed that classifying extreme outliers generally produce better results for all traits except for Neuroticism and Openness.
