We present a machine learning approach to distinguish texts translated to Chinese (by humans) from texts originally written in Chinese, with a focus on a wide range of syntactic features. Using Support Vector Machines (SVMs) as classifier on a genre-balanced corpus in translation studies of Chinese, we find that constituent parse trees and dependency triples as features without lexical information perform very well on the task, with an F-measure above 90\%, close to the results of lexical n-gram features, without the risk of learning topic information rather than translation features. Thus, we claim syntactic features alone can accurately distinguish translated  from original Chinese. Translated Chinese exhibits an increased use of determiners, subject position pronouns, NP + ``的'' as NP modifiers, multiple NPs or VPs conjoined by ``、'', among other structures. We also interpret the syntactic features with reference to previous translation studies in Chinese, particularly the usage of pronouns.
