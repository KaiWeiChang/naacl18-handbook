End-to-end neural models show great promise towards building conversational agents that are trained from data and on-line experience using supervised and reinforcement learning. However, these models require a large corpus of dialogues to learn effectively. For goal-oriented dialogues, such datasets are expensive to collect and annotate, since each task involves a separate schema and database of entities. Further, the Wizard-of-Oz approach commonly used for dialogue collection does not provide sufficient coverage of salient dialogue flows, which is critical for guaranteeing an acceptable task completion rate in consumer-facing conversational agents. In this paper, we study a recently proposed approach for building an agent for arbitrary tasks by combining dialogue self-play and crowd-sourcing to generate fully-annotated dialogues with diverse and natural utterances. We discuss the advantages of this approach for industry applications of conversational agents, wherein an agent can be rapidly bootstrapped to deploy in front of users and further optimized via interactive learning from actual users of the system.
