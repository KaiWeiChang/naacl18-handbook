The ARRAU corpus is an anaphorically annotated corpus of English providing rich linguistic information about anaphora resolution. The most distinctive feature of the corpus is the annotation of a wide range of anaphoric relations, including bridging references  and discourse deixis in addition to identity (coreference). Other distinctive  features include treating all NPs as markables, including non-referring NPs; and the annotation of a variety of morphosyntactic and semantic mention and entity attributes, including the genericity status of the entities referred to by markables. The corpus however has not been extensively used for anaphora resolution research so far. In this paper, we discuss three datasets extracted from the ARRAU  corpus to support the three subtasks of the CRAC 2018 Shared Task--identity anaphora resolution over ARRAU-style markables, bridging references resolution, and discourse deixis; the evaluation scripts assessing system performance on those datasets; and preliminary results on these three tasks that may serve as baseline for subsequent research in these phenomena.
