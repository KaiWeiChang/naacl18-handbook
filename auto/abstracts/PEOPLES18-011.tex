Most scholarly works in the field of computational detection of humour derive their inspiration from the incongruity theory. Incongruity is an indispensable facet in drawing a line between humorous and non-humorous occurrences but is immensely inadequate in shedding light on what actually made the particular occurrence a funny one. Classical theories like Script-based Semantic Theory of Humour and General Verbal Theory of Humour try and achieve this feat to an adequate extent. In this paper we adhere to a more holistic approach towards classification of humour based on these classical theories with a few improvements and revisions. Through experiments based on our linear approach and performed on large data-sets of jokes, we are able to demonstrate the adaptability and show componentizability of our model, and that a host of classification techniques can be used to overcome the challenging problem of distinguishing between various categories and sub-categories of jokes.
