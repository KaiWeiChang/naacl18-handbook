In this paper we introduce the notion of Demand-Weighted Completeness, allowing estimation of the completeness of a knowledge base with respect to how it is used. Defining an entity by its classes, we employ usage data to predict the distribution over relations for that entity. For example, instances of person in a knowledge base may require a birth date, name and nationality to be considered complete. These predicted relation distributions enable detection of important gaps in the knowledge base, and define the required facts for unseen entities. Such characterisation of the knowledge base can also quantify how usage and completeness change over time. We demonstrate a method to measure Demand-Weighted Completeness, and show that a simple neural network model performs well at this prediction task.
