Code-switching is a phenomenon of mixing grammatical structures of two or more languages under varied social constraints. The code-switching data differ so radically from the benchmark corpora used in NLP community that the application of standard technologies to these data degrades their performance sharply. Unlike standard corpora, these data often need to go through additional processes such as language identification, normalization and/or back-transliteration for their efficient processing. In this paper, we investigate these indispensable processes and other problems associated with syntactic parsing of code-switching data and propose methods to mitigate their effects. In particular, we study dependency parsing of code-switching data of Hindi and English multilingual speakers from Twitter. We present a treebank of Hindi-English code-switching tweets under Universal Dependencies scheme and propose a neural stacking model for parsing that efficiently leverages the part-of-speech tag and syntactic tree annotations in the code-switching treebank and the preexisting Hindi and English treebanks. We also present normalization and back-transliteration models with a decoding process tailored for code-switching data. Results show that our neural stacking parser is 1.5\% LAS points better than the augmented parsing model and 3.8\% LAS points better than the one which uses first-best normalization and/or back-transliteration.
