The Common European Framework of Reference (CEFR) guidelines describe language proficiency of learners on a scale of 6 levels. While the description of CEFR guidelines is generic across languages, the development of automated proficiency classification systems for different languages follow different approaches. In this paper, we explore universal CEFR classification using  domain-specific and domain-agnostic, theory-guided as well as data-driven features. We report the results of our preliminary experiments in monolingual, cross-lingual, and multilingual classification with three languages: German, Czech, and Italian. Our results show that both monolingual and multilingual models achieve similar performance, and cross-lingual classification yields lower, but comparable results to monolingual classification.
