\section{Keynote Address: Dilek Hakkani-T\"{u}r}
\index{Hakkani-T\"{u}r, Dilek}

\begin{center}
\begin{Large}
{\bfseries\Large Google Assistant or My Assistant? Towards Personalized Situated Conversational Agents}\vspace{1em}\par
\end{Large}

\daydateyear, 9:00--10:10 \vspace{1em}\\
\sessionchair{Heng}{Ji}
\PlenaryLoc \\
\vspace{1em}\par
\end{center}

\noindent
{\bfseries Abstract:} Interacting with machines in natural language has been a holy grail since the beginning of computers. Given the difficulty of understanding natural language, only in the past couple of decades, we started seeing real user applications for targeted/limited domains. More recently, advances in deep learning based approaches enabled exciting new research frontiers for end-to-end goal-oriented conversational systems. However, personalization (i.e., learning to take actions from users and learning about users beyond memorizing simple attributes) remains a research challenge. In this talk, I’ll review end-to-end situated dialogue systems research, with components for situated language understanding, dialogue state tracking, policy, and language generation. The talk will highlight novel approaches where dialogue is viewed as a collaborative game between a user and an agent in the presence of visual information. The situated conversational agent can be bootstrapped using user simulation (crawl), improved through interactions with crowd-workers (walk), and iteratively refined with real user interactions (run).

\vspace{2ex}\centerline{\rule{.5\linewidth}{.5pt}}\vspace{2ex}
\setlength{\parskip}{1ex}\setlength{\parindent}{0ex}

% \vspace{3em}\par 
% \vfill

\noindent
{\bfseries Biography:} Dilek is a research scientist at Google Research Dialogue Group and has previously held positions at Microsoft Research, ICSI, and AT\&T Labs – Research. She is a fellow of the IEEE and of ISCA. Her research interests include conversational AI, natural language and speech processing, spoken dialogue systems, and machine learning for language processing.

\newpage
