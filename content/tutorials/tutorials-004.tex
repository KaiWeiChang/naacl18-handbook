
\begin{tutorial}
  {The interplay between lexical resources and Natural Language Processing}
  {naacl2018-tutorials-008}
  {\daydateyear, \tutorialafternoontime}
  {\TutLocD}
\end{tutorial}

 Incorporating linguistic, world and common sense knowledge into AI/NLP systems is currently an important research area, with several open problems and challenges. At the same time, processing and storing this knowledge in lexical resources is not a straightforward task. We propose to address these complementary goals from two methodological perspectives: the use of NLP methods to help the process of constructing and enriching lexical resources and the use of lexical resources for improving NLP applications. This tutorial may be useful for two main types of audience: those working on language resources who are interested in becoming acquainted with automatic NLP techniques, with the end goal of speeding and/or easing up the process of resource curation; and on the other hand, researchers in NLP who would like to benefit from the knowledge of lexical resources to improve their systems and models.

\vspace{2ex}\centerline{\rule{.5\linewidth}{.5pt}}\vspace{2ex}
\setlength{\parskip}{1ex}\setlength{\parindent}{0ex}

  {\bfseries Jose Camacho-Collados} (collados@di.uniroma1.it, http://www.josecamachocollados.com) is a Google Doctoral Fellow and PhD student at Sapienza University of Rome. His research focuses on Natural Language Processing and, more specifically, on the area of lexical and distributional semantics. Jose has experience in utilizing lexical resources for NLP applications, while enriching and improving these resources by extracting and processing knowledge from textual data. On this area he is co-organizing the upcoming SemEval 2018 shared task on Hypernym Discovery. Previously, he co-organized a workshop on “Sense, Concept and Entity Representations and their Applications” at EACL 2017 and a tutorial on the same topic at ACL 2016. His background education includes an Erasmus Mundus Master in Natural Language Processing and Human Language Technology and a 5-year BSc degree in Mathematics.
%  \index{Camacho-Collados, Jose}

  {\bfseries Luis Espinosa Anke} (espinosa-ankel@cardiff.ac.uk, http://www.luisespinosa.net) received his BA in English Philology in 2006 (Univ. of Alicante, Spain), and his PhD in Natural Language Processing in 2017 (Univ. Pompeu Fabra, Spain). He holds two MAs, one in English-Spanish Translation (Univ. of Alicante), and an Erasmus Mundus MA in Natural Language Processing (NLP) (Univ. of Wolverhampton and Univ. Aut\`{o}noma de Barcelona). His research interests lie in the intersection between structured representations of knowledge and NLP, specifically computational lexicography and distributional semantics. He is a co-organizer of the upcoming SemEval 2018 shared tasks on Hypernym Discovery and Multilingual Emoji Prediction. Previously, he co-organized the Spanish NLP conference (2014), the Focused NER task at the Open Knowledge Extraction challenge at ESWC (2017).
%  \index{Anke, Luis Espinosa}

  {\bfseries Mohammad Taher Pilehvar} (mp792@cam.ac.uk, http://people.ds.cam.ac.uk/mp792/) is a research associate at the University of Cambridge. Taher’s research lies in lexical semantics, mainly focusing on semantic representation and similarity. In the past, he has co-instructed three tutorials on these topics (EMNLP 2015, ACL 2016, and EACL 2017) and co-organised three SemEval tasks. He has also co-authored several conference (including two ACL best paper nominations, at 2013 and 2017) and journal papers, including different semantic representation techniques based on heterogeneous lexical resources.
%  \index{Pilehvar, Mohammad Taher}
