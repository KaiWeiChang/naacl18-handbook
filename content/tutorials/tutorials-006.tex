\begin{tutorial}
  {Deep Learning for Conversational AI}
  {naacl2018-tutorials-033}
  {\daydateyear, \tutorialafternoontime}
  {\TutLocF}
\end{tutorial}

Spoken Dialogue Systems (SDS) have great commercial potential as they promise to revolutionise the way in which humans interact with machines. The advent of deep learning led to substantial developments in this area of NLP research, and the goal of this tutorial is to familiarise the research community with the recent advances in what some call the most difficult problem in NLP

From a research perspective, the design of spoken dialogue systems provides a number of significant challenges, as these systems depend on: a) solving several difficult NLP and decision-making tasks; and b) combining these into a functional dialogue system pipeline. A key long-term goal of dialogue system research is to enable open-domain systems that can converse about arbitrary topics and assist humans with completing a wide range of tasks. Furthermore, such systems need to autonomously learn on-line to improve their performance and recover from errors using both signals from their environment and from implicit and explicit user feedback. While the design of such systems has traditionally been modular, domain and language-specific, advances in deep learning have alleviated many of the design problems.

The main purpose of this tutorial is to encourage dialogue research in the NLP community by providing the research background, a survey of available resources, and giving key insights to application of state-of-the-art SDS methodology into industry-scale conversational AI systems. We plan to introduce researchers to the pipeline framework for modelling goal-oriented dialogue systems, which includes three key components: 1) Language Understanding; 2) Dialogue Management; and 3) Language Generation. The differences between goal-oriented dialogue systems and chat-bot style conversational agents will be explained in order to show the motivation behind the design of both, with the main focus on the pipeline SDS framework. For each key component, we will define the research problem, provide a brief literature review and introduce the current state-of-the-art approaches. Complementary resources (e.g. available datasets and toolkits) will also be discussed. Finally, future work, outstanding challenges, and current industry practices will be presented. All of the presented material will be made available online for future reference.

\vspace{2ex}\centerline{\rule{.5\linewidth}{.5pt}}\vspace{2ex}
\setlength{\parskip}{1ex}\setlength{\parindent}{0ex}

  {\bfseries Pei-Hao (Eddie) Su} (eddysu@poly-ai.com, https://eddy0613.github.io/) is a co-founder and CTO of PolyAI, a London-based startup looking to use the latest developments in NLP to create a general machine learning platform for deploying spoken dialogue systems. He holds a PhD from the Dialogue Systems group, University of Cambridge, where he worked under the supervision of Professor Steve Young. His research interests centre on applying deep learning, reinforcement learning and Bayesian approaches to dialogue management and reward estimation, with the aim of building systems that can learn directly from human interaction. He has given several invited tallks at academia and industry such as Apple, Microsoft, General Motor and DeepHack.Turing. He received the best student paper award at ACL 2016.
  \index{Su, Pei-Hao}

  {\bfseries Nikola Mrk\v{s}i\'{c}} (nikola@poly-ai.com, http://mi.eng.cam.ac.uk/~nm480) a co-founder and CEO of PolyAI, a London-based startup looking to use the latest developments in NLP to create a general machine learning platform for deploying spoken dialogue systems. He holds a PhD from the Dialogue Systems group, University of Cambridge, where he worked under the supervision of Professor Steve Young. His research is focused on belief tracking in human-machine dialogue, specifically in moving towards building open-domain, cross-lingual language understanding models that are fully data-driven. He is also interested in deep learning, semantics, Bayesian nonparametrics, unsupervised and semi-supervised learning. He previously gave a tutorial on word vector space specialisation at EACL 2017, and will teach a course on the same topic at ESSLLI 2018. He also gave invited talks at the REWORK AI Personal Assistant summit and the Chatbot Summit.
\index{Mrkšić, Nikola}


  {\bfseries I\~{n}igo Casanueva}  (inigo@poly-ai.com, http://mi.eng.cam.ac.uk/~ic340/) is a Machine Learning engineer at PolyAI, a London-based startup looking to use the latest developments in NLP to create
a general machine learning platform for deploying spoken dialogue systems. He got his PhD from the University of Sheffield and later he worked as Research Assistant in the Dialogue Systems group, University of Cambridge. His main research interest focuses on increasing the scalability of machine learning based dialogue management, looking for methods to make deep learning and/or reinforcement learning applicable to real world dialogue management tasks. He has published several papers on the topic, two of them nominated to best paper award.
%  \index{Casanueva, I\~{n}igo}

  {\bfseries Ivan Vuli\'{c}} (iv250@cam.ac.uk, https://sites.google.com/site/ivanvulic/) is a Senior Research Associate in the Language Technology Lab at the University of Cambridge. He holds a PhD from KU Leuven, obtained summa cum laude. Ivan is interested in representation learning, human language understanding, distributional, lexical, and multi-modal semantics in monolingual and multilingual contexts, and transfer learning for enabling cross-lingual NLP applications. He co-lectured a tutorial on monolingual and multilingual topic models and applications at ECIR 2013 and WSDM 2014, a tutorial on word vector space specialisation at EACL 2017, and a tutorial on cross-lingual word representations at EMNLP 2017. He will lecture a course on word vector space specialisation at ESSLLI 2018. He has given invited talks at academia and industry such as Apple Inc., University of Cambridge, UCL, University of Copenhagen, Paris-Saclay, and Bar-Ilan University.
%  \index{Vuli\'{c}, Ivan}


