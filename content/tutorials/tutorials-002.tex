\begin{tutorial}
  {Deep Learning Approaches to Text Production}
  {tutorial-002}
  {\daydateyear, \tutorialmorningtime}
  {\TutLocB}
\end{tutorial}

Text production is a key component of many NLP applications. In data-driven approaches, it is used for instance, to generate dialogue turns from dialogue moves, to verbalise the content of Knowledge bases or to generate natural English sentences from rich linguistic representations, such as dependency trees or Abstract Meaning Representations. In text-driven methods on the other hand, text production is at work in sentence compression, sentence fusion, paraphrasing, sentence (or text) simplification, text summarisation and end-to-end dialogue systems.

Following the success of encoder-decoder models in modeling sequence-rewriting tasks such as machine translation, deep learning models have successfully been applied to the various text production tasks. In this tutorial, we will cover the fundamentals and the state-of-the-art research on neural models for text production. Each text production task raises a slightly different communication goal (e.g, how to take the dialogue context into account when producing a dialogue turn; how to detect and merge relevant information when summarising a text; or how to produce a well-formed text that correctly capture the information contained in some input data in the case of data-to-text generation). We will outline the constraints specific to each subtasks and examine how the existing neural models account for them.

\vspace{2ex}\centerline{\rule{.5\linewidth}{.5pt}}\vspace{2ex}
\setlength{\parskip}{1ex}\setlength{\parindent}{0ex}

  {\bfseries Claire Gardent} (claire.gardent@loria.fr, https://members.loria.fr/CGardent/) is a research scientist at CNRS (the French National Center for Scientific Research). Prior to joining the CNRS, she worked at the Universit\'{e}}́ de Clermont-Ferrand, Saarbr\"{u}cken Universit\"{a}t and Amsterdam Universiteit. She received her Ph.D. degree from the University of Edinburgh. Her research interests include (executable) semantic parsing, natural language generation and simplification and, more recently, the use of computational linguistics for linguistic analysis. She was nominated Chair of the EACL and acted as program chair for various international conferences, workshops and summer schools (EACL, ENLG, SemDIAL, SIGDIAL, ESSLLI, *SEM). She currently heads the WebNLG project (Nancy, Bolzano, Stanford SRI) and is the chair of SIGGEN, the ACL Special Interest Group in Natural Language Generation. Recently she co-organised the WebNLG Shared Task, a challenge on generating text from RDF data.
  \index{Gardent, Claire}
 
  {\bfseries Shashi Narayan} (shashi.narayan@ed.ac.uk, http://homepages.inf.ed.ac.uk/snaraya2/) is a research associate at the School of Informatics at the University of Edinburgh. His research focuses on natural language generation, understanding and structured predictions. A major aim of his research is to build on the hypothesis that tailoring a model with knowledge of the task structure and linguistic requirements, such as syntax and semantics, leads to a better performance. The questions raised in his research are relevant to various natural language applications such as question answering, paraphrase generation, semantic and syntactic parsing, document understanding and summarization, and text simplification. He mostly rely on machine learning techniques such as deep learning and spectral methods to develop NLP frameworks. His research has appeared in computational linguistics journals (e.g., TACL, Computational Linguistics and Pattern Recognition Letters) and in conferences (e.g., ACL, EMNLP, COLING, EACL and INLG). He was nominated on the SIGGEN board (2012-14) as a student member. He co-organised the WebNLG Shared Task, a challenge on generating text from RDF data. Recently, he is nominated as an area co-chair for Generation at NAACL HLT 2018.
  \index{Narayan, Shashi}
