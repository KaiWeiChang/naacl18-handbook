\begin{tutorial}
  {Modelling Natural Language, Programs, and their Intersection}
  {naacl2018-tutorials-014}
  {\daydateyear, \tutorialmorningtime}
  {\TutLocA}
\end{tutorial}

As computers and information grow a more integral part of our world, it is becoming more and more important for humans to be able to interact with their computers in complex ways. One way to do so is by programming, but the ability to understand and generate programming languages is a highly specialized skill. As a result, in the past several years there has been an increasing research interest in methods that focus on the intersection of programming and natural language, allowing users to use natural language to interact with computers in the complex ways that programs allow us to do. In this tutorial, we will focus on machine learning models of programs and natural language focused on making this goal a reality. First, we will discuss the similarities and differences between programming and natural language. Then we will discuss methods that have been designed to cover a variety of tasks in this field, including automatic explanation of programs in natural language (code-to-language), automatic generation of programs from natural language specifications (language-to-code), modeling the natural language elements of source code, and analysis of communication in collaborative programming communities. The tutorial will be aimed at NLP researchers and practitioners, aiming to describe the interesting opportunities that models at the intersection of natural and programming languages provide, and also how their techniques could provide benefit to the practice of software engineering as a whole.

\vspace{2ex}\centerline{\rule{.5\linewidth}{.5pt}}\vspace{2ex}
\setlength{\parskip}{1ex}\setlength{\parindent}{0ex}

  {\bfseries Graham Neubig} (gneubig@cs.cmu.edu, http://phontron.com) is an assistant professor at Carnegie Mellon University specializing in natural language processing and machine learning. One of his major research interests is models that link together natural language and code, including summarizing the intent of code in natural language, generating code from natural language, or discovering the correspondences between the two modalities. He has previously given well-attended tutorials at NLP conferences (EMNLP and YRSNLP) and the Lisbon Machine Learning Summer School, and has won a number of best papers (e.g. EMNLP2016 and EACL2017) and given invited talks, including an upcoming one on this topic at the AAAI Workshop on NLP for Software Engineering.
  %\index{Neubig, Graham}

  {\bfseries Miltiadis Allamanis} (miallama@microsoft.com, https://miltos.allamanis.com) is a researcher at Microsoft Research, Cambridge, UK at the Deep Program Understanding project. He is researching applications of machine learning and natural language processing to software engineering and programming languages to create smart software engineering tools for developers. Miltos has published in both machine learning and software engineering conferences and is an author of a recent survey on machine learning for source code (https://ml4code.github.io). He received his PhD at the University of Edinburgh, UK advised by Dr. Charles Sutton.
  %\index{Allamanis, Miltiadis}

