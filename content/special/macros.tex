% Macros for the production of Conference Handbooks / Program Brochures from
% ACLPUB bundles from the START conference manager.

\newlength{\w}      % width of the space available for paper entries in a
                    % multi-track schedule
\newlength{\tsl}    % with of a time specification in a schedule
\newlength{\en}     % \en-space
%\newlength{\tmplen}

\setlength{\tabcolsep}{.5ex} 

\newenvironment{SingleTrackSchedule}%
{%
  \settowidth{\en}{--}
  \setlength{\w}{\linewidth}%
  \settowidth{\tsl}{$\,$} % temporary abuse of this length measure
  \addtolength{\w}{-2\tsl}%
  \settowidth{\tsl}{00:00}
  \addtolength{\w}{-2\tsl}%
  \addtolength{\w}{-2\tabcolsep}%
  \addtolength{\w}{-1\en}%
  \begin{longtable}{@{}%
      >{\raggedleft}p{\tsl}%
      @{$\,$}%
      p{1\en}%
      @{$\,$}%
      >{\raggedright}p{\tsl}%
      >{\RaggedRight}p{\w}@{}}%
}{%
  \end{longtable}%
}

\newenvironment{TwoTrackSchedule}{
\settowidth{\en}{--}
\setlength{\w}{\linewidth}%
\settowidth{\tsl}{$\,$}
\addtolength{\w}{-2\tsl}%
\settowidth{\tsl}{\footnotesize 00:00am}
\addtolength{\tsl}{2ex}
\addtolength{\w}{-2\tsl}%
\addtolength{\w}{-2\tabcolsep}%
\addtolength{\w}{-1\en}%
\addtolength{\w}{-1cm}%
\renewcommand{\arraystretch}{1.1}
%\setlength{\tmplen}{\tabcolsep}
%\addtolength{\tmplen}{-.5\arrayrulewidth}
\begin{longtable}{@{}|%
    >{\raggedleft}p{\tsl}%
    @{$\,$}%
    p{1\en}%
    @{$\,$}%
    >{\raggedright}%
    p{\tsl}|%
    >{\centering\arraybackslash}p{.5\w}|%
    >{\centering\arraybackslash}p{.5\w}|@{}}%
}{\end{longtable}}

\newenvironment{ThreeTrackSchedule}{
\settowidth{\en}{--}
\setlength{\w}{\linewidth}%
\settowidth{\tsl}{$\,$}
\addtolength{\w}{-2\tsl}%
\settowidth{\tsl}{00:00am}
\addtolength{\tsl}{2ex}
\addtolength{\w}{-2\tsl}%
\addtolength{\w}{-2\tabcolsep}%
\addtolength{\w}{-1\en}%
\addtolength{\w}{-1cm}%
\renewcommand{\arraystretch}{1.1}
%\setlength{\tmplen}{\tabcolsep}
%\addtolength{\tmplen}{-.5\arrayrulewidth}
\begin{longtable}{@{}%
    >{\raggedleft}p{\tsl}%
    @{$\,$}%
    p{1\en}%
    @{$\,$}%
    >{\raggedright}%
    p{\tsl}|%
    >{\centering\arraybackslash}p{.333\w}|%
    >{\centering\arraybackslash}p{.333\w}|%
    >{\centering\arraybackslash}p{.333\w}@{}}%
}{\end{longtable}}

\newenvironment{FourTrackSchedule}{
\settowidth{\en}{--}
\setlength{\w}{\linewidth}%
\settowidth{\tsl}{$\,$}
\addtolength{\w}{-2\tsl}%
\settowidth{\tsl}{00:00am}
\addtolength{\tsl}{2ex}
\addtolength{\w}{-2\tsl}%
\addtolength{\w}{-2\tabcolsep}%
\addtolength{\w}{-1\en}%
\addtolength{\w}{-1cm}%
\renewcommand{\arraystretch}{1.1}
%\setlength{\tmplen}{\tabcolsep}
%\addtolength{\tmplen}{-.5\arrayrulewidth}
\begin{longtable}{@{}|%
    >{\raggedleft}p{\tsl}%
    @{$\,$}%
    p{1\en}%
    @{$\,$}%
    >{\raggedright}%
    p{\tsl}|%
    >{\centering\arraybackslash}p{.25\w}|%
    >{\centering\arraybackslash}p{.25\w}|%
    >{\centering\arraybackslash}p{.25\w}|%
    >{\centering\arraybackslash}p{.25\w}|@{}}%
}{\end{longtable}}

%%%%%%%%%%% BreakTime Macro %%%%%%%%%%%%%%%%%%%%%%%%%%%%%%%%%%%%%%%%%%%%%%
\newcommand{\BreakTime}[4]{%
% adds a gray background to a break or a break-style event
% #1 start time
% #2 end   time
% #3 number of parallel tracks
% #4 label of the break or break-style event
\multicolumn{3}{c}{\cellcolor[gray]{.8}} & 
\multicolumn{#3}{c}{\cellcolor[gray]{.8}} \\[-3ex]\hline 
\bfseries #1 & -- & \bfseries #2 &
\multicolumn{#3}{c|}{\bfseries #4}}
%%%%%%%%%%%%%%%%%%%%%%%%%%%%%%%%%%%%%%%%%%%%%%%%%%%%%%%%%%%%%%%%%%%%%%%%%%

%%%%%%%%%%% PlenaryEvent Macro %%%%%%%%%%%%%%%%%%%%%%%%%%%%%%%%%%%%%%%%%%%%%%
\newcommand{\PlenaryEvent}[4]{%
% event with nothing else going on in parallel
\bfseries #1 & -- & \bfseries #2 &
\multicolumn{#3}{>{\centering\arraybackslash}p{\w}|@{}}{\bfseries #4}}
%%%%%%%%%%%%%%%%%%%%%%%%%%%%%%%%%%%%%%%%%%%%%%%%%%%%%%%%%%%%%%%%%%%%%%%%%%

%%%%%%%%%%% SessionHeader Macro %%%%%%%%%%%%%%%%%%%%%%%%%%%%%%%%%%%%%%%%%%%%%%
\newcommand{\SingleTrackSessionHeader}[3]{%
% event with nothing else going on in parallel
\ifthenelse{\equal{{#1}}{{}}}{&}{#1 & -- } & #2 
\ifthenelse{\equal{#1}{{}}}{}{\hfill}
&\multicolumn{1}{>{\raggedright\arraybackslash}m{\w}}{\bfseries #3}}
%%%%%%%%%%%%%%%%%%%%%%%%%%%%%%%%%%%%%%%%%%%%%%%%%%%%%%%%%%%%%%%%%%%%%%%%%%

% insert references to the page ranges 
\newcommand{\ppp}[1]{pp. \pageref{#1start}--\pageref{#1end}}

% #1 = last name
% #2 = last name (initials)
% #3 = first name
% #4 = first name (initials)
% #5 = name prefix, a.k.a. 'von part'
% #6 = name prefix (initials)
% #7 = name affix, a.k.a. 'junior part'
% #8 = name affix (initials)


% the following code requires the biblatex package
\DeclareNameFormat{authorswithinitials}{%
  % name format that prints the list of authors with initals
  \ifcase\value{uniquename}%
%%%    \usebibmacro{name:given-family}{#1}{#4}{#5}{#7}%
    \nameparts{#1}%
      \usebibmacro{name:given-family}
        {\namepartfamily}
        {\namepartgiveni}
        {\namepartprefix}
        {\namepartsuffix}%
  \or
    \ifuseprefix
    \nameparts{#1}%
      \usebibmacro{name:given-family}
        {\namepartfamily}
        {\namepartgiveni}
        {\namepartprefix}
        {\namepartsuffixi}%
      \usebibmacro{name:given-family}
        {\namepartfamily}
        {\namepartgiveni}
        {\namepartprefixi}
        {\namepartsuffixi}%
%%%      {\usebibmacro{name:given-family}{#1}{#4}{#5}{#8}}
%%%      {\usebibmacro{name:given-family}{#1}{#4}{#6}{#8}}%
  \or
%%%    \usebibmacro{name:given-family}{#1}{#4}{#5}{#7}%
\nameparts{#1}%
      \usebibmacro{name:given-family}
        {\namepartfamily}
        {\namepartgiveni}
        {\namepartprefix}
        {\namepartsuffix}%
  \fi
  \usebibmacro{name:andothers}}

\DeclareNameFormat{authorlastnames}{%
  % name format that prints the list of author last names
  \ifcase\value{uniquename}%
%%%    \usebibmacro{name:family}{#1}{#4}{#5}{#7}%
\nameparts{#1}%
      \usebibmacro{name:given-family}
        {\namepartfamily}
        {\namepartgiveni}
        {\namepartprefix}
        {\namepartsuffix}%
  \or
    \ifuseprefix
\nameparts{#1}%
      \usebibmacro{name:given-family}
        {\namepartfamily}
        {\namepartgiveni}
        {\namepartprefix}
        {\namepartsuffixi}%
\nameparts{#1}%
      \usebibmacro{name:given-family}
        {\namepartfamily}
        {\namepartgiveni}
        {\namepartprefixi}
        {\namepartsuffixi}%
%%%      {\usebibmacro{name:family}{#1}{#4}{#5}{#8}}
%%%      {\usebibmacro{name:family}{#1}{#4}{#6}{#8}}%
  \or
\nameparts{#1}%
      \usebibmacro{name:given-family}
        {\namepartfamily}
        {\namepartgiveni}
        {\namepartprefix}
        {\namepartsuffix}%
%%%    \usebibmacro{name:family}{#1}{#4}{#5}{#7}%
  \fi
  \usebibmacro{name:andothers}}

\DeclareNameFormat{fullauthornames}{%
  % name format that prints the list of authors with full author names
  \ifcase\value{uniquename}%
  \nameparts{#1}%
      \usebibmacro{name:given-family}
        {\namepartfamily}
        {\namepartgiven}
        {\namepartprefix}
        {\namepartsuffix}%
%%%    \usebibmacro{name:given-family}{#1}{#3}{#5}{#7}%
    \or
    \ifuseprefix
\nameparts{#1}%
      \usebibmacro{name:given-family}
        {\namepartfamily}
        {\namepartgiven}
        {\namepartprefix}
        {\namepartsuffixi}%
\nameparts{#1}%
      \usebibmacro{name:given-family}
        {\namepartfamily}
        {\namepartgiven}
        {\namepartprefixi}
        {\namepartsuffixi}%
%%%        {\usebibmacro{name:given-family}{#1}{#3}{#5}{#8}}
%%%        {\usebibmacro{name:given-family}{#1}{#3}{#6}{#8}}%
  \or
\nameparts{#1}%
      \usebibmacro{name:given-family}
        {\namepartfamily}
        {\namepartgiven}
        {\namepartprefix}
        {\namepartsuffix}%
%%%  \usebibmacro{name:given-family}{#1}{#3}{#5}{#7}%
  \fi
  \usebibmacro{name:andothers}}

% insert the list of authors with first name initials
\DeclareCiteCommand{\citeauthorslastnamesonly}{%
  \boolfalse{citetracker}%
  \boolfalse{pagetracker}%
  \usebibmacro{prenote}%
}{\ifciteindex{\indexnames{labelname}}{}%
  \printnames[authorlastnames]{labelname}%
}{\multicitedelim}{\usebibmacro{postnote}}

% insert the list of authors with first name initials
\DeclareCiteCommand{\citeauthorswithinitials}{%
  \boolfalse{citetracker}%
  \boolfalse{pagetracker}%
  \usebibmacro{prenote}%
}{\ifciteindex{\indexnames{labelname}}{}%
  \printnames[authorswithinitials]{labelname}%
}{\multicitedelim}{\usebibmacro{postnote}}
  
% insert the list of authors with full names
\DeclareCiteCommand{\citefullauthornames}{%
  \boolfalse{citetracker}%
  \boolfalse{pagetracker}%
  \usebibmacro{prenote}%
}{%\ifciteindex{\indexnames{labelname}}{}%
  \indexnames{labelname}%
  \printnames[fullauthornames]{labelname}%
}{\multicitedelim}{\usebibmacro{postnote}}
                     
\DeclareFieldFormat[inproceedings]{citetitle}{#1}

\newcommand{\paperauthor}[1]{{\em #1}}
\newcommand{\papertitle}[1]{\citetitle{#1}}

% insert an entry into a multi-track schedule cell
\newcommand{\paperentry}[1]{%
  \renewcommand{\baselinestretch}{.8}%
  \setlength{\parindent}{0pt}%
    \begin{small}%
      \parbox[t]{\linewidth}{%
        {\em \papertitle{#1}}
        \vspace{.5ex}}\par%
      \vfill
      \parbox[b]{\linewidth}{\raggedright%
        \paperauthor{\citeauthorslastnamesonly{#1}}%
        %% \mbox{}~\dotfill~ {\bfseries p.~\pageref{#1}}
        }%
    \end{small}%
}

\newcommand{\sempaperentry}[1]{%
  &&$\bullet$&
  \parbox[t]{\linewidth}{\raggedright\papertitle{#1}\newline
  {\itshape\paperauthor{\citefullauthornames{#1}}}
\mbox{}~\dotfill~\pageref{#1}}}

\newcommand{\atpaperentry}[2]{%
  \footnotesize\parbox[t]{\linewidth}{\noindent%
    \makebox[0pt][r]{#1\hspace*{\tabcolsep}}{%
      \paperauthor{\citeauthorswithinitials{#2}}}:
    \papertitle{#2}\mbox{}~\dotfill{p.~\pageref{#2}\par}}}

% insert an entry into a poster session index
\newcommand{\posterentry}[1]{%
  \renewcommand{\baselinestretch}{1.2}%
  \settowidth{\w}{\bfseries p.~\pageref{#1}}%
  \addtolength{\w}{1em}%
  \parbox[t]{\linewidth-\w}{\noindent\raggedright{\bfseries\papertitle{#1}}%
    \linebreak[0]\ {---~{\paperauthor{\citeauthorswithinitials{#1}}}}\mbox{}~\dotfill\makebox[0pt][l]%
    {\makebox[\w][r]{\bfseries p.~\pageref{#1}}}\vspace{1em}\par}}

% insert an abstract into a list of abstracts (for posters, no time needed)
\newcommand{\posterabstract}[1]{%
  \noindent%
  \begin{minipage}{\linewidth}%
    \label{#1}%
          {\bfseries\normalsize\papertitle{#1}}\\
          \normalsize\paperauthor{\citefullauthornames{#1}}
  \end{minipage}\vspace{1mm}\\\nopagebreak%
  \noindent{\small\input{auto/abstracts/#1}}\par}

% insert an abstract into a list of abstracts
\newcommand{\paperabstract}[5]{%
  % #1 day
  % #2 time
  % #3 session title
  % #4 location
  % #5 paper id
  \noindent%
  \begin{minipage}{\linewidth}%
    \label{#5}%
      %\rule{\linewidth}{1pt}\linebreak
          {\bfseries\normalsize\papertitle{#5}} \\
          \hfill\normalsize\paperauthor{\citefullauthornames{#5}}
%      {\small #1 #2 --- #4}\vspace{.5ex}\\
          \hfill {\small #2}
%    \parbox[t]{.8\linewidth}{\centering
%      {\bfseries\papertitle{\citetitle{#5}}}\linebreak
%      \paperauthor{\citefullauthornames{#5}}}%
%    \parbox[t]{.2\linewidth}{\raggedleft\small%
%      #1\linebreak #2\linebreak #4}\par\nopagebreak
  %% \begin{center}\label{#5}%
  %%   \sloppy\hyphenpenalty=0%
  %%   %{\small --- Session: #3 --- \vspace{.5em}}\linebreak		% Redundant info? - B
  %%   {\bfseries \citetitle{#5}} \vspace{.5em}\linebreak
  %%   {\itshape    \citefullauthornames{#5}}%\vspace{.5em}\linebreak	% Reduce space between header and abstract - B
  %%   %#1 #2 --- #4\linebreak 						% Redundant info? - B
  \end{minipage}\vspace{1mm}\\\nopagebreak%
  \noindent{\small\input{auto/abstracts/#5}}\par}

\newcommand{\TutorialCoverPage}[4]{%
\vspace*{.25in}
% #1 Tutorial Number
% #2 Tutorial Title
% #3 Tutorial Presenter
% #4 Tutorial Presenter Information
%\begin{minipage}{\linewidth}
\begin{centering}
\includegraphics[height=7.5cm,clip=true]{content/fmatter/conference-logo.eps}\\

\vspace{.5cm}

{\Large
The 2012 Conference of the \linebreak
North American Chapter of the \linebreak
Association for Computational Linguistics: \linebreak
Human Language Technologies}\par

\vspace{3cm}

%\begin{tabular}{@{}>{\centering\arraybackslash}p{.7\linewidth}@{}}
\centerline{\huge\bfseries Tutorial~#1\vspace{.25em}}
{\Large Tutorial Notes}\\

\vspace{1cm}

\begin{minipage}{.7\linewidth}
\centering\Large%\setstretch{1.1}
{\bfseries#2}
\end{minipage}

\vfill

{\itshape\Large #3}\\[.5em]
{#4}\\
\vspace{1cm}

{June 3, 2012}\\
{Montr\'{e}al, Queb\'{e}c, Canada}\\\end{centering}
%\end{minipage}
\clearpage}

\newcommand{\STARSEM}{$^\ast$SEM}

\newcommand{\nix}{\cellcolor{mygray}}
% used for the room occupation tables in the local information section

\newcommand{\mrup}[2]{\multirow{#1}{*}{%
%\rotatebox[origin=c]{90}{#2}}%
\begin{sideways}#2\end{sideways}%
}}

\newcommand{\PaperInSchedule}[4][false]{%
  % #1 (optional) with or without a \pageref to the abstract
  % #2 start time (can be empty)
  % #3 end time (can be empty)
  % #4 paper id
  \ifthenelse{\equal{{#2}}{{}}}{&&\hfill}{#2&--&} % ... start time
  \ifthenelse{\equal{{#3}}{{}}}{$\bullet$}{#3}    %   ... end time
    & \parbox[t]{\linewidth}{\raggedright\papertitle{#4}\newline
      {\paperauthor{\citefullauthornames{#4}}}%
      \ifthenelse{\equal{#1}{true}}{\mbox()~\dotfill~\pageref{#4}}}}

\newcommand{\ScheduleItem}[4][false]{%
  % #1 (optional) with or without a \pageref to the abstract
  % #2 start time (can be empty)
  % #3 end time (can be empty)
  % #4 paper id
  \ifthenelse{\equal{{#2}}{{}}}{&&\hfill}{#2&--&} % ... start time
  \ifthenelse{\equal{{#3}}{{}}}{$\bullet$}{#3}    %   ... end time
    & \parbox[t]{\linewidth}{\raggedright\papertitle{#4}\newline
      {\paperauthor{\citefullauthornames{#4}}}%
      \ifthenelse{\equal{#1}{true}}{\mbox()~\dotfill~\pageref{#4}}}}

\newenvironment{singledaywsprogram}[5]{%
% #1 worshop id
% #2 workshop number
% #3 workshop title
% #4 workshop day
% #5 workshop venue
\section{\textbf{W~#2:} #3}\label{#1}

\begin{center}
{\bfseries\Large #4}\vspace{1em}\par
{\itshape Venue:} #5\vspace{0.75em}\par
{\large\bfseries Program\vspace{0.75em}}\par
\setlength{\w}{\linewidth}
\begin{SingleTrackSchedule}}%
{\end{SingleTrackSchedule}\end{center}}

%% WORKSHOP SCHEDULE %%%%%%%%%%%%%%%%%%%%%%%%%%%%%%%%%%%%%%%%%%%%

%\newlength{\wsscheduleindentation}
%\newlength{\wsschedulelabelwidth}
\newenvironment{wsschedule}[5]{%
  % #1 workshop title
  % #2 workshop number
  % #3 workshop label
  % #4 workshop paper ID
  % #5 workshop location
  %\addcontentsline{toc}{section}{{{\bfseries W~#2:} #4}}

  %\addcontentsline{toc}{section}{{{\bfseries W~#2:} 
  %    \ifthenelse{\equal{#1}{none}}{#4}{#1}}}

  % For workshops numbered 0, don't prepend W0 on the contents page
  \clearpage
  \ifthenelse{\equal{#2}{0}}
%%             {\section{\textbf{#1}}}
             {\section{#1}}
%%
%% For ACL 2017, we do not show the Wx prefix and instead only show the title
%%             {\section[{\bfseries W#2:} #1]{{\bfseries Workshop #2:} #1}}
             {\section{#1}}
             %% {{\addcontentsline{toc}{section}{{#1}}}}
             %% {{\addcontentsline{toc}{section}{{{\bfseries W#2:} #1}}}}

  \markboth{}{}
  \label{#3}
  \begin{center}
%    {\huge{\Large Workshop #2:}\vspace{.5ex}\\
    %% {\Large
    %%   \ifthenelse{\equal{#2}{0}}{{}}{{\huge{Workshop #2:\\\vspace{.5ex}}}}
    %%   #1\par}
    {\ifthenelse{\equal{#4}{0}}{{}}{{Organizers: \paperauthor{\citefullauthornames{#4}}}\par}}
    {\large Venue: #5 \vspace{2em}}\\
    %% {\huge Workshop Program\vspace{1ex}}
  \end{center}
  \markright{{\bfseries W~#2:} #1}{}%
  \begin{list}{{}}{%
      \settowidth{\labelwidth}{\bfseries 12:00pm--12:00pm}
      \setlength{\labelsep}{1ex}
      \setlength{\topsep}{0pt}
      \setlength{\parsep}{0pt}
      \setlength{\listparindent}{0pt}
      \setlength{\itemsep}{.5ex}
      \setlength{\rightmargin}{0pt}
      \setlength{\leftmargin}{\labelwidth}
      \addtolength{\leftmargin}{\labelsep}}
}{\end{list}}

\newcounter{WorkshopCounter}

\newenvironment{tutorial}[4]{%
  \stepcounter{TutorialCounter}
  % #1 short title
  % #2 tutorial bibtex ID
  % #3 date and time
  % #4 location

  % Cannot get \papertitle to expand
  \section[\textbf{T\theTutorialCounter:} {#1}]{Tutorial \theTutorialCounter}
  \begin{center}
    \begin{Large}
      \bfseries \papertitle{#2}\\ \vspace{2em}\par
    \end{Large}
    {\itshape \tutorialauthors{#2}}\vspace{1em}\par
    #3 \vspace{1em}\\
    #4
  \end{center}}

\newcounter{TutorialCounter}

\newcommand{\tutorialauthors}[1]{
    \paperauthor{\citefullauthornames{#1}}}

\newcommand{\wspaperentry}[1]{
  \parbox[t]{\linewidth}{\raggedright\papertitle{#1}\newline
    \paperauthor{\citefullauthornames{#1}}}}

\newcommand{\sessionchair}[2]{
  \emph{Chair: #1 #2}\par\index{#2, #1}}

\newcommand{\papertitleandauthor}[1]{
  {\bfseries \papertitle{#1}}\\
  \paperauthor{\citefullauthornames{#1}}}

\newcommand{\papertableentry}[1]{
  \scriptsize
  {\papertitle{#1}}\newline
  {\paperauthor{\citeauthorslastnamesonly{#1}}}}

\newenvironment{SessionOverview}[7]{
  \section[#1]{#1 Overview -- #2}
  \setheaders{#1}{#2}
  \begin{center}
    \righthyphenmin2
    \sloppy
    \begin{tabular}{>{\RaggedRight}p{0.8in}|>{\RaggedRight}p{0.8in}|>{\RaggedRight}p{0.8in}|>{\RaggedRight}p{0.8in}|>{\RaggedRight}p{0.8in}}
      \bf Track A & \bf Track B & \bf Track C & \bf Track D & \bf Track E \\
      \it #3 & \it #4 & \it #5 & \it #6 & \it #7 \\
      \TrackALoc & \TrackBLoc & \TrackCLoc & \TrackDLoc & \TrackELoc \\
      \hline\hline}
{\end{tabular}\end{center}}

\newenvironment{TwoSessionOverview}[4]{
  \section[#1]{#1 Overview -- #2}
  \setheaders{#1}{#2}
  \begin{center}
    \righthyphenmin2
    \sloppy
    \begin{tabular}{|>{\RaggedRight}p{2.0in}|>{\RaggedRight}p{2.0in}|}
      \hline
      \bf Track A & \bf Track B \\\hline
      \it #3 & \it #4 \\
      \TrackALoc & \TrackBLoc \\
      \hline\hline}
{\hline\end{tabular}\end{center}}

\newenvironment{ThreeSessionOverview}[5]{
  \section[#1]{#1 Overview -- #2}
  \setheaders{#1}{#2}
  \begin{center}
    \righthyphenmin2
    \sloppy
    \begin{tabular}{|>{\RaggedRight}p{1.3in}|>{\RaggedRight}p{1.3in}|>{\RaggedRight}p{1.3in}|}
      \hline
      \bf Track A & \bf Track B & \bf Track C \\\hline
      \it #3 & \it #4 & \it #5 \\
      \TrackALoc & \TrackBLoc & \TrackCLoc \\
      \hline\hline}
{\hline\end{tabular}\end{center}}

\newcommand{\daydate}{DAY, DATE}
\newcommand{\daydateyear}{DAY, DATE, YEAR}
\newcommand{\setdaydateyear}[3]{
  \renewcommand{\daydateyear}{#1, #2, #3\xspace}
  \renewcommand{\daydate}{#1, #2\xspace}}
